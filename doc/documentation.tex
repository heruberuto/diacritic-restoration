\documentclass[12pt]{article}
\usepackage[margin=1in]{geometry}
\usepackage{amsmath,amsthm,amssymb}
\usepackage{titling}
\usepackage{url}
\usepackage{dirtree}
\setlength{\droptitle}{-5em}

\newcommand{\N}{\mathbb{N}}
\newcommand{\R}{\mathbb{R}}
\newcommand{\E}{\mathbb{E}}
\newcommand{\sub}[1] {\noindent{\textbf{{#1)}}}}
\setcounter{section}{0}

\begin{document}
    \title{CoLi -- Final Project -- Diacritics~Restoration}
    \author{Herbert~Ullrich~(\textbf{2576412})}
    \maketitle
    \section{Introduction}
    Diacritic restoration is the task to restore diacritics, novel algorithms based on recurrent neural networks
    such as~\cite{naplava}
    \subsection{Used corpora}
    We have decided to use the \textit{Corpus for training and evaluating diacritics restoration systems}~\cite{corpus} proposed
    in 2018 as the standard training and testing set for the DR task.
    That way, our results will get easy to compare with the state-of-the-art methods, such as~\cite{naplava}
    \subsection{Project structure}
    \dirtree{%
    .1 /.
    .2 bin.
    .2 home.
    .3 jeancome.
    .4 texmf.
    .5 tex.
    .6 latex.
    .7 dirtree.
    .3 jeancomeson.
    .3 jeancomedaughter.
    .2 usr.
    .3 bin.
    .3 games.
    .4 fortunes.
    .3 include.
    .3 local.
    .4 bin.
    .4 share.
    .5 texmf.
    .6 fonts.
    .6 metapost.
    .6 tex.
    .3 share.
    }
    \bibliography{references}
    \bibliographystyle{plain}
\end{document}
